\section{Welcome to \emph{Kubuntu}}\label{welcome-to-kubuntu}

\emph{Kubuntu} is a user-friendly Linux-based operating system that use
the \emph{KDE} \emph{Plasma Desktop}. With a predictable six-month
release cycle as part of the \emph{Ubuntu} project, \emph{Kubuntu} is
the Linux distribution for everyone.

\emph{Kubuntu} includes more than 1,000 default packages (applications
or utilities) and has access to more than 64,000 other packages to suit
your needs. \emph{Kubuntu} is based on the Linux kernel and includes the
core \emph{Ubuntu} applications as well as \emph{KDE} software.
\emph{Kubuntu}'s core applications include software for the most common
needs, like:

\begin{itemize}
\itemsep1pt\parskip0pt\parsep0pt
\item
  Browsing the web.
\item
  Personal Information Manager which includes email.
\item
  Office applications
\item
  Playing multimedia files.
\item
  Plus much more!
\end{itemize}

\subsection{The \emph{Kubuntu} Philosophy}\label{the-kubuntu-philosophy}

\begin{itemize}
\itemsep1pt\parskip0pt\parsep0pt
\item
  \emph{Kubuntu} will always use the solid base of the \emph{Ubuntu}
  project, plus the latest from the \emph{KDE} project. As part of the
  Ubuntu project and community, \emph{Kubuntu} will continue to use the
  infrastructure and support that the \emph{Ubuntu} project offers. We
  will strive to be the best \emph{KDE} based Linux distribution
  available.
\item
  \emph{Kubuntu} will always be free of charge. There is no extra fee
  for an enterprise edition; we make our best work available to everyone
  on the same free terms.
\item
  \emph{Kubuntu} will always include the best translations and
  accessibility infrastructure that the free software community has to
  offer, to make \emph{Kubuntu} usable by as many people as possible.
\item
  \emph{Kubuntu} will always be committed to the principles of free
  software and open source development; we shall encourage people to use
  free and open source software, improve it, and pass it on.
\end{itemize}

\subsection{What is Linux?}\label{what-is-linux}

Linux is an operating system kernel that resembles the Unix operating
system. The kernel is the main software required for any operating
system, providing a communication bridge between hardware and software.
Linux has become a leading element of the worldwide movement to embrace
free and open source software. The term ``GNU/Linux'' is a way of
referring to operating systems based on the Linux kernel combined with
parts from the .. \_GNU Project: \url{http://www.gnu.org}.

\subsection{What is \emph{KDE}?}\label{what-is-kde}

\emph{KDE} is an international technology community that creates and
supports free software for desktop and portable computing. Among
\emph{KDE}'s products are a modern desktop system for Linux and Unix
platforms, comprehensive office productivity and groupware suites, as
well as hundreds of software applications in various categories
including internet and web applications, multimedia, entertainment,
education, graphics, and software development. \emph{KDE} software is
translated into more than 65 languages and is built for ease of use with
modern accessibility principles in mind. \emph{KDE}'s full-featured
applications run natively on Linux, BSD, Solaris, Windows, and Mac OS X.
The \emph{KDE} Workspace is the default desktop for \emph{Kubuntu}.

\subsection{Thank You!}\label{thank-you}

The entire \emph{Kubuntu} team thanks you for choosing \emph{Kubuntu!}

\begin{description}
\item[Authors]
Kubuntu Docs Team
\item[Version]
0.1 of 5/1/2015
\end{description}
 
